\chapter{Évaluation de mon apport}

En choisissant de modifier les données fournies pour la réalisation du « Netflix Prize », je change un peu le paradigme de base. Ainsi, il ne faut plus uniquement se baser sur la création d'un groupe d’utilisateurs avec des notes similaires mais bien de faire coïncider les préférences des utilisateurs selon leurs genres préférés. 

\vspace{5mm}

Une comparaison de mon approche avec celle obtenue par Netflix ou les vainqueurs du concours est compliquée au vu des différences entre les données utilisées. L’ajout d’un genre pour les films entraine une complexité supplémentaire à la problématique de base. Ainsi, si les données utilisées pour la réalisation du « Netflix Prize » avaient contenu le genre de chaque film cela aurait ajouté une difficulté supérieure, des calculs supplémentaires via les algorithmes mais aussi des temps d'accès supplémentaires à la donnée pour délivrer le résultat. Dans ces conditions, il est facile de dire que les algorithmes utilisés et les méthodes appliquées pour résoudre la problèmatique du Netflix Prize auraient été drastiquements différentes, impactant par la même occasion le résultat obtenu. 

\vspace{5mm}

Evaluer un système de recommandation permet de déterminer si ses performances sont reconnues vis-à-vis de l'objectif de départ, de la concurrence ou de l'existant (un système de recommdation déjà en place). Ici, le système appliqué repose sur des données différentes et ne permet pas d'évaluer sa performance vis-à-vis de la concurrence et de l'existant. 

\vspace{5mm}

Mon implémentation reprend en partie la problématique du  « Netflix Prize » avec une donnée supplémentaire, le genre du film associé. Mon implémentation se présente comme une solution dans un cas très spécifique. Il s’agit du cas où plusieurs voisins ont une distance identique et permet donc d’apporter une information permettant de choisir le voisin le plus fiable pour créer une recommandation pertinente. Il s’agit d’une situation extrêmement spécifique, reproductible que dans de rares cas. Malgré le degré de complexité supplémentaire, ma solution permet de mettre en évidence une recommandation plus pertinente dans le cas des voisins similaires mais avec des goûts différents. Cette approche permet de répondre positivement à cette nouvelle problématique. Cependant, ma réponse à ce problème entraine un coût supplémentaire en termes de temps et de complexité. Il serait important de se poser la question du coût de cette implémentation à plus grande échelle, sur un nombre de données beaucoup plus important et couplé à d’autres algorithmes comme le fait Netflix ou les différents participants du « Netflix Prize ». L’efficacité de ma solution peut être considéré comme néfaste et négligeable par rapport au nombre de cas où dans la réalité ma solution pourrait permettre une meilleure recommandation. De plus, les données fournies par Netflix son issues d’un choix réfléchi permettant de remettre en cause la pertinence d’avoir un genre associé à chaque film pour en faire une recommandation efficace. 


