\chapter*{Introduction}
\addcontentsline{toc}{chapter}{Introduction}
\markboth{Introduction}{Introduction}
\label{chap:introduction}




\vspace{5mm}

Le "Dayli mix" de Spotify, les vidéos recommandées sur la page d’accueil YouTube ou encore les suggestions d’abonnement sur Instagram, ces recommandations sont exclusivement basées sur vos "j'aime" ou interactions précédentes. Ces recommandations permanentes sont issues d’un système de recommandation, aujourd’hui très présent dans notre quotidien et dans beaucoup de services que nous utilisons.

\vspace{5mm}

Aujourd’hui, une entreprise mise énormément sur les systèmes de recommandation : \textbf{Netflix}. L'idée système de recommandation performant est rapidement devenue un besoin pour l'entreprise et ce dès la création de Netflix. Wilmot Reed Hastings est un entrepreneur américain et fondateur de Netflix. En 1997, la location de film est courante et comme un grand nombre de personne à cette époque il oublie de rendre la copie d'un film. Au bout de six semaines, la boutique lui demande de payer une quarantaine de dollars de frais de retard\supercite{NetflixOrigins}. Ainsi, l'idée de Netflix est née :  une formule par abonnement avec la possibilité de conserver la copie d'un film aussi longtemps que souhaité le tout par internet. Aucune boutique physique mais un catalogue complet sur internet, plus complet que n'importe quelle boutique existante ausssi grande soit-elle et la copie du film choisie est envoyée par la poste. L'idée est brillante mais devant l'immensité du catalogue, les clients choisissent les mêmes films : les gros succès du cinéma. L'idée est alors pour l'entreprise de mieux  connaître les goûts de ses utilisateurs à partir de leurs statistiques de consommation. Les titres loués sont utilisés pour proposer, des titres moins populaires permettant à Netflix de faire tourner les bluckbusters et d'envoyer les titres les moins populaires. Ce système de recommandation est appelé  \textbf{CineMatch} et il est présenté comme l’un des atous de Netflix. 

\vspace{5mm}


En 2018, plus de 30 millions de nouveaux abonnés ont rejoint le service de SVOD\footnote{Subscription Video On Demand : Permet d'accéder avec un abonnement payant à un catalogue de vidéo à la demande généralement sans publicité, en illimité et sans engagement.} de Netflix , portant à environ 140 millions le nombre de comptes payants \footnote{Le premier mois de souscription étant gratuit, ils ne sont pas comptabilisé} soit plus de deux fois la population Française. D'après les chiffres du Figaro \supercite{FigaroChiffre} en 5 ans, Netflix comptabilise en avril 2019 déjà 5 millions d'abonnés et a donc dépassé en terme d'abonnés payants,  l'acteur historique CANAL+ qui comptabilise sur la même période 4,757 millions d'abonnés individuels en France. 

\vspace{5mm}

Premièrement, il est necessaire de connaître qui en France consommme Netflix, selon une enquête d'Harris Interractive\footnote{Harris Interractive est un institut d'études marketing et de sondages d'opinion (Politique - Digital - Corporate).}
et NPA\footnote{NPA Conseil est cabinet de conseil stratégique et opérationnel au service de la transformation numérique.} qui dresse le profil type d'un utilisateur de SVOD en France\supercite{NPA}. En moyenne,  25\% des utilisateurs d'un service de SVOD utilisent se service quotidiennement pour une session durant en moyenne 1 à 3h. La tranche horaire la plus utilisée est celle du "prime time" (de 21h à 23h). Dans cette enquête, deux faits sont marquants, premièrement, 63\% des utilisateurs d'un service de SVOD en France sont  âgés de moins de 35 ans tandis que 64\% des spectateurs de la "télévision classique" sont âgés de plus de 50 ans. Malgré cette différence d'âge, les utilisateurs de SVOD privilégient tout de même majoritairement le téléviseur pour regarder leurs contenus.


\vspace{5mm}


Autrefois, sans réels concurrents, \supercite{NetflixNumerama} de nombreux acteurs en voyant le succès de Netflix ont décidés de rentrer dans la course, aujourd'hui NetFlix se trouve face à Amazon Prime Video et Hulu mais prochainement de nombreux géants vont rejoindre le marché comme Disney+, Paramount+ et bien plus encore. 

\vspace{5mm}



%https://www.francetvinfo.fr/culture/series/netflix/enquete-franceinfo-comment-netflix-sy-prend-pour-nous-rendre-accros_3189939.html



Chaque utilisateur donne à son insu des informations à Netflix sur ses habitudes et les exploites. Ce qu'on regarde, sur quel support et à quel moment de la journée, permet par la suite de recommander du contenu qui serait suceptible de nous plaire. Aujourd'hui, la majorité des contenus consommés sur la platforme de Netflix sont issus de son système de recommadation. 