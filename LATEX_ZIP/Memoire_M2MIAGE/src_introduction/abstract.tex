\begin{abstract}
\hskip7mm
\begin{spacing}{1.3}


Dans un avenir très proche, les systèmes de recommandation seront devenus incontournables, non seulement dans l’audiovisuel, mais globalement sur internet et peu importe leurs types de contenus. Demain, il n’est pas impossible que des sites tels que LinkedIn recommandent à des employeurs le parfait candidat pour être embauché. 

\vspace{5mm}

Actuellement, une entreprise mise beaucoup sur les systèmes de recommandation : Netflix.  Ce service, utilisé par des millions d’utilisateurs dans le monde met en avant son système de recommandation dans son plan de communication mais estime aussi qu’il s’agit d’un élément primordial pour garder un de ses utilisateurs sur sa plateforme. 

\vspace{5mm}

En premier lieu, nous étudierons les systèmes de recommandation de manière général et l’approche que Netflix a appliqué sur sa plateforme. 

\vspace{5mm}

Dans un second temps, nous répondrons aux questions suivantes : Quelles sont les différentes contraintes des systèmes de recommandation ? Quelles sont leurs limites ? Comment Netflix à limité les contraintes ? Et quelle est l’importance de la recommandation pour Netflix ? 

\vspace{5mm}

Dans une troisième partie, nous dresserons un état de l’art en se basant sur le Netflix Prize permettant d’en apprendre d'avantage sur le fonctionnement technique de la recommandation par Netflix.

\vspace{5mm}

Finalement, je présenterai ma propre interprétation du Netflix Prize en y proposant une solution. Cette dernière sera schématisée et expliquée afin de déterminer sa pertinence. 




\end{spacing}
\end{abstract}