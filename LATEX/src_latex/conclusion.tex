\chapter*{Conclusion}
\addcontentsline{toc}{chapter}{Conclusion}
\markboth{Conclusion}{Conclusion}
\label{sec:conclusion}




Aujourd’hui, les systèmes de recommandation sont déjà extrêmements présents dans notre quotidien. 

\vspace{5mm}


Il semblerait que l’une des approches à envisager pour un grand nombre d’acteur du web soit une adaptation aux systèmes de recommandation. Un grand nombre de personnes ont déjà l’habitude d’utiliser ces systèmes aux quotidiens. Pour une entreprise, c’est un moyen efficace d’adapter son contenu face à chaque client permettant ainsi d’apporter une expérience unique et satisfaisante. Un internaute satisfait est un client potentiel pour chaque acteur du web désirant faire fructifier son business et son chiffre d’affaire. 

\vspace{5mm}

C’est dans ce but que Netflix a introduit Cinematch, avec un nombre limité de DVD pour le dernier blockbuster à la mode, il ne fallait pas frustrer les abonnés et toujours avoir une recommandation suffisamment intéressante pour pallier cette pénurie de DVD. 
Au moment de sa digitalisation, les besoins de Netflix en matière de recommandation ont évolué. Aujourd’hui, 80\% des contenus consommés sur la plateforme sont issus d’une recommandation, preuve de son importance au sein de son business. Avec un catalogue grossissant de mois en mois, un nombre d’utilisateurs toujours en forte hausse et des concurrents toujours plus nombreux, il est devenu primordial pour Netflix d’être toujours plus efficace et c'est pour cela que la plateforme à décider d’utiliser la recommandation à outrance. C’est dans cette logique que la plateforme a lancé en 2007 la compétition « Netflix Prize » essayant avec le minimum d’information d’apporter la recommandation la plus pertinente possible. Cette compétition a permis une avancé formidable pour le monde de la recherche en terme de recommandation en mettant en évidence la force de certains algorithmes. Le rayonnement du « Netflix Prize » a été une grande publicité positive pour l’entreprise avec un gain double : le nombre d’abonné à la plateforme à drastiquement augmenté tandis que sa recommandation s’est affiné grâce aux travaux des différents participants.

\vspace{5mm}

Pour finir, ce mémoire m'a permis d’étudier les systèmes de recommandation de Netflix sous différents aspects. Me permettant ainsi de comprendre fonctionnement de la recommandation autant sur le plan marketing que technique, son but et son importance pour l'entreprise et ses abonnés. J'ai moi-même essayé de mettre en place un algorithme efficace permettant de faire une recommandation pertinente dans un cas extrême. Preuve qu'il existe encore beaucoup d'améliorations et de recherches possibles autour de la recommandation afin de rendre les utilisateurs toujours plus accros aux différents services. 
 
