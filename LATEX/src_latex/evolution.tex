\chapter{Évolutions possibles}



Dans ce chapitre, je vais décrire les évolutions possibles concernant le  « Netflix Prize »  puis celles de Netflix et enfin celles de mon apport.

\vspace{5mm}

Concernant le « Netfliz Prize », l’équipe BellKor’s Pragmatic Chaos est la première équipe à avoir remporté le Netlflix Prize challenge avec un RMSE meilleur que l’algorithme Cinematch.  Ce concours a permis une avancée considérable pour des systèmes de recommandation mettant en avant la puissance de la factorisation matricielle pour le filtrage collaboratif mais aussi l’émergence de nouveaux algorithmes pour le machine learning. Netflix a autorisé l’utilisation des données mise à disposition pour la recherche, ce qui permet encore actuellement des avancées et des recherches sur l’approche des systèmes de recommandation et leur évolutivité. De nombreuses personnes déposent leurs solutions sur des outils de partages collaboratifs comme GitHub par exemple. Le Netflix Prize ne permettra plus de générer autant d’enthousiasme de la part des chercheurs et passionnés qu’entre 2006 à 2009. Cependant, il n’est pas impossible qu’un nouvel algorithme issu de ce challenge bouscule le monde de la recommandation. 


\vspace{5mm}


Au sujet de Netflix, avec l’arrivée de nouveaux concurrents, la plateforme doit perpétuellement se renouveler. Une grande partie de son budget est utilisé pour l’acquisition de droit sur des contenus, comme pour la série "Friend" mais aussi pour la création de contenus originaux. La création de contenus originaux est un choix logique et raisonné. En effet, de nombreux acteurs prennent leur independance comme Disney, ce qui induit une perte des droits pour Netflix de beaucoup de licence forte comme l’ensemble des films PIXAR. A court terme, il est logique d’imaginer que Netflix va utiliser ses points forts pour garder ses utilisateurs et en acquérir de nouveaux. Comme démontré tout au long de ce mémoire, la grande force de Netflix est représentée par son système de recommandation à partir duquel 80\% du contenu consommé provient. Il est logique de penser que Netflix va utiliser son système de recommandation pour mettre en avant des contenus exclusifs à la plateforme. Logiquement, les futures évolutions de Netflix côté software se baseront sur l’interface mais principalement sur l’amélioration continue de leur algorithme de recommandation permettant ainsi de diminuer petit à petit la marge d’erreur du système de recommandation bien quelle soit aujourd’hui déjà négligeable. 


\vspace{5mm}

Pour finir, concernant mon apport, il existe plusieurs axes d’améliorations. Premièrement, une amélioration de la fiabilité et la qualité de mes algorithmes permettrait de pousser la prédiction plus loin. Une prédiction plus fiable permet de produire une note pratiquement aussi fiable que celle donné explicitement par l’utilisateur et donc d’être utilisée pour prédire d’autres films. Par exemple, si un utilisateur n’a jamais regardé de Comédie et qu'une première prédiction a mis en évidence une prédiction fiable pour un film Comique, une seconde étape pourrait être l’exploitation de cette note pour recommander d’autres films dans le genre Comédie. 

\vspace{5mm}

Secondement, j’ai essayé de démontrer que prendre en compte une seconde donnée tel que le genre du film permettait d’apporter une information pertinente et ce malgré la complexité supplémentaire. Je suis convaincu que l’utilisation de sous-genre comme les Comédies Romantiques permettrait d’augmenter la pertinence de la recommandation. L’utilisation de l’année du film serait sûrement une autre voie à explorer, permettant ainsi de déterminer si l’utilisateur préfère les films anciens ou plus récents. 

\vspace{5mm} 

Rétrospectivement en analysant mon travail, j’ai appliqué deux fois l’algorithme des plus proches voisins. Il serait sûrement plus pertinent d’utiliser plusieurs algorithmes sur plusieurs dimensions comme l’équipe gagnante du « Netflix Prize ». Je pense que l’agrégat des recommandations issu de plusieurs algorithmes offre une force supplémentaire afin d’obtenir une recommandation plus qualitative et plus pertinente. 
